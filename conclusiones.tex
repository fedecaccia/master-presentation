\section{Conclusiones}

\subsection*{Logros alcanzados}
\normalsize
\begin{frame}
{Conclusiones}
{Logros alcanzados}

\begin{itemize}
\item <2-> Análisis en detalle de algunas porciones de subsistemas considerando la dinámica global del problema.
\item <3-> Aceleración de los tiempos de cálculo.
\item <4-> Implementación eficiente de la técnica de acoplamiento desarrollada que permitió acoplar ParGPFEP, RELAP5, Fermi y PUMA, entre otros.
\item <5-> Estudios en sistemas con bajas y altas cantidades de incógnitas.
\item <6-> Aportes al análsis del SSP del RA-10.
\item <7-> En general el método de $Broyden$ arrojó los mejores resultados.
\item <8-> Generalización de la técnica de acoplamiento a cálculos de dinámica de núcleo.
\item <9-> Los estudios en acoplamiento neutrónico-termohidráulico nos permiten realizar la siguiente propuesta: 
El método del \textit{Punto fijo} acelera el tiempo de cálculo ($\sim$ 35\%) respecto del comúmente utilizado método de $Picard$.
\end{itemize}

\end{frame}

\subsection*{Trabajos futuros}

\begin{frame}
{Conclusiones}
{Trabajos futuros}

\begin{itemize}
\item <2-> Estudios de dinámica de núcleo acoplando mayor cantidad de modelos
\item <3-> Estudio de la evolución de la criticidad en el núcleo del RA-10 ante la acción del SSP
\item <4-> Diferenciación automática
\end{itemize}

\end{frame}
